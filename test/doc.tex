\documentclass[
  fontsize=12pt  %
 ,headinclude    %
 ,headsepline    % line between head an document text
%,BCOR=12mm      %
]{scrbook}       % twosided, A4 paper
%]{article}       % twosided, A4 paper

\usepackage{tabularx}
\usepackage[table, dvipsnames]{xcolor}
\usepackage[skins,listings,breakable]{tcolorbox}
\usepackage[T1]{fontenc}
\usepackage{graphicx}
\usepackage{tocbasic}
\usepackage{scrhack}
\usepackage{hyperref}

\makeatletter
\patchcmd{\@classz}
  {\CT@row@color}
  {\oldCT@column@color}
  {}
  {}
\patchcmd{\@classz}
  {\CT@column@color}
  {\CT@row@color}
  {}
  {}
\patchcmd{\@classz}
  {\oldCT@column@color}
  {\CT@column@color}
  {}
  {}
\makeatother

\definecolor{gray1}{gray}{0.92}
\definecolor{gray2}{gray}{0.88}
\definecolor{gray3}{gray}{0.85}
\definecolor{yelight}{rgb}{.9,.8,.6}

\begin{document}
\setcounter{chapter}{0}
\setcounter{section}{0}

\chapter{Package cpp2latex}
\section{Summary}
\begin{flushleft}
This package \texttt{cpp2latex} converts a \texttt{C++} file into
\LaTeX {} code, to be included in a \LaTeX {} document.

It creates a box with a title and does some syntax highlighting;
a canonical ``Hello World'' \texttt{C++} file
looks like this:

\end{flushleft}

\begin{table}[ht]
  \rowcolors{0}{gray!15}{gray!5}
  \begin{center}
    \scalebox{1.1}{
      \tcbox[title=main.cc] {
        \input{test/main.cc.tex}
      }
    }
  \end{center}
  \caption{main.cc}
\end{table}

\section{Usage}

%\begin{flushleft}
You include the generated file into your \LaTeX {} document like this:\newpage
\begin{verbatim}
\begin{table}[ht]
  \rowcolors{0}{gray!15}{gray!5}
  \begin{center}
    \scalebox{1.1}{
      \tcbox[title=main.cc] {
        \input{test/main.cc.tex}
      }
    }
  \end{center}
  \caption{main.cc}
\end{table}
\end{verbatim}

And you generate the file like this:\\

\begin{verbatim}
cat test/main.cc | bin/cpp2latex > test/main.cc.tex
\end{verbatim}

%\end{flushleft}

\section{Build}
\begin{flushleft}

This package depends on reflex, a modern \texttt{C++} replacement of gnu flex.
See \newline \href{https://www.genivia.com/doc/reflex/html/}
{\texttt{https://www.genivia.com/doc/reflex/html/}} \newline

Please check/adjust the \texttt{Makefile} to reflect how ``reflex'' is installed
on your system.\\[1em]

To build the package, you use make:\\[1em]

{\verb
make
}\\[2em]

You can also generate an example pdf file with:\\[1em]

{\verb
make example
}\\[1em]

This requires that ``pdflatex'' is installed.\\[1em]

it ``installs'' the binary in \texttt{/bin/cpp2latex}; if you wish you may run\\[1em]

{\verb
make install
}\\[1em]

creates a symlink in \texttt{\textbackslash{}usr\textbackslash{}local\textbackslash{}bin}.
\end{flushleft}

\chapter{Examples}

\begin{table}[ht]
  \rowcolors{0}{gray!15}{gray!5}
  \begin{center}
    \scalebox{1.1}{
      \tcbox[title=Euler14.h] {
        \input{test/Euler14.h.tex}
      }
    }
  \end{center}
  \caption{Euler14.h}
\end{table}


\begin{table}[ht]
  \rowcolors{0}{gray!15}{gray!5}
  \begin{center}
    \scalebox{1.0}{
      \tcbox[title=bench.cc] {
        \input{test/bench.cc.tex}
      }
    }
  \end{center}
  \caption{bench.cc}
\end{table}

\begin{table}[ht]
  \rowcolors{0}{gray!15}{gray!5}
  \begin{center}
    \scalebox{0.5}{
      \tcbox[title=dbinit.cc] {
        \input{test/dbinit.cc.tex}
      }
    }
  \end{center}
  \caption{dbinit.cc}
\end{table}

\end{document}


